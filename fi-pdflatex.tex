\documentclass[oneside,
  digital, %% This option enables the default options for the
           %% digital version of a document. Replace with `printed`
           %% to enable the default options for the printed version
           %% of a document.
  table,   %% Causes the coloring of tables. Replace with `notable`
           %% to restore plain tables.
  nolof,     %% Prints the List of Figures. Replace with `nolof` to
           %% hide the List of Figures.
  nolot,     %% Prints the List of Tables. Replace with `nolot` to
           %% hide the List of Tables.
  %% More options are listed in the user guide at
  %% <http://mirrors.ctan.org/macros/latex/contrib/fithesis/guide/mu/fi.pdf>.
]{fithesis3}
%% The following section sets up the locales used in the thesis.
\usepackage[resetfonts]{cmap} %% We need to load the T2A font encoding
\usepackage[T1,T2A]{fontenc}  %% to use the Cyrillic fonts with Russian texts.
\usepackage[
  main=english, %% By using `czech` or `slovak` as the main locale
                %% instead of `english`, you can typeset the thesis
                %% in either Czech or Slovak, respectively.
  german, russian, czech, slovak %% The additional keys allow
]{babel}        %% foreign texts to be typeset as follows:
%%
%%   \begin{otherlanguage}{german}  ... \end{otherlanguage}
%%   \begin{otherlanguage}{russian} ... \end{otherlanguage}
%%   \begin{otherlanguage}{czech}   ... \end{otherlanguage}
%%   \begin{otherlanguage}{slovak}  ... \end{otherlanguage}
%%
%% For non-Latin scripts, it may be necessary to load additional
%% fonts:
\usepackage{paratype}
\def\textrussian#1{{\usefont{T2A}{PTSerif-TLF}{m}{rm}#1}}
%%
%% The following section sets up the metadata of the thesis.
\thesissetup{
    university    = mu,
    faculty       = fi,
    type          = mgr,
    author        = Martin Štefanko,
    gender        = m,
    advisor       = {Bruno Rossi, PhD},
    title         = {Use of Transactions within a Reactive Microservices Environment},
    TeXtitle      = {Use of Transactions within a Reactive Microservices Environment},
    keywords      = {transactions, Narayana, JTA, reactive, microservices, asynchronous, saga, compensating transactions},
    TeXkeywords   = {transactions, Narayana, JTA, reactive, microservices, asynchronous, saga, compensating transactions},
}
\thesislong{abstract}{
abstract
}
\thesislong{thanks}{
   thanks
}
%% The following section sets up the bibliography.
\usepackage{booktabs}

\usepackage{makeidx}      %% The `makeidx` package contains
\makeindex                %% helper commands for index typesetting.
%% These additional packages are used within the document:
\usepackage{paralist}
\usepackage{amsmath}
\usepackage{amsthm}
\usepackage{amsfonts}
\usepackage{url}
\usepackage{menukeys}


\begin{document}
\chapter{Introduction}

\chapter{Transaction concepts}

This chapter introduces the basic notions of transactions, their properties and common problems of the management of transactions across multiple nodes in the distributed systems. 

\section{Transaction}

A transaction is an unit of processing that provides all-or-nothing property to the work that is conducted within its scope, also ensuring that shared resources are protected from multiple users \cite{java_tran_processing}. It represents an unified and inseparable sequence of operations that are either all provided or none of them take effect. 

From the application point of view there exist several transaction models in which the transactions can be executed. The applicable models in the transaction management are local, programmatic and declarative transaction models. All three models will be described in detail in the following section.

The transaction can end in two forms: it can be either \textit{commited} or \textit{aborted}. The commit determines the successful outcome - all operations within the transaction have been performed and their results are permanently stored in a durable storage. The abort means that all performed operations have been undone and the system is in the same state as if the transaction have not been started.

Generally the achieving of above features may differ. The most common pattern for the transaction processing is a two phase commit protocol with the ACID transactions. Other approaches are based on the relaxation of the one or more of ACID properties to adjust to the real world environments.

\section{2PC protocol}

\section{ACID properties}

A transaction can be viewed as a group of business logic statements with certain shared properties \cite{nar_wf}. Generally considered properties are one or more of atomicity, consistency, isolation and durability. These four properties are often referenced as ACID properties \cite{haerder_reuter_1983} and they describe the major points important for the transaction concepts.

\subsection{Atomicity}

The transaction consists of a sequence of operations performed on different resources or by different participants. Atomicity means that all operation in the transaction are performed as if they were a single operation. When the transaction commits successfully all of its participants are also required to perform a valid commit. Conversely, if the transaction fails and is aborted all performed operations and effects are forced to be undone. This defines a possibility to abort at any point so that all changes done by the transaction will be reverted to the state before the transaction start.

Atomicity is generally achieved by the usage of the consensus multi-phase protocols. Standardized protocol is the two phase commit protocol which is used by the majority of modern transaction systems. 

\subsection{Consistency}

Consistency describes that the transaction maintains the consistency of the system and resources that it is performed on. When the transaction is started on the consistent system this system must remain consistent when the transaction ends - it moves from one consistent state to another.

Unlike other transactional properties (A, I, D), consistency cannot be realized by the transaction system as it does not hold any semantic knowledge about the resources it manipulates \cite{java_tran_processing}. Therefore achieving this property is the responsibility of the application code.

\subsection{Isolation}

An isolation property takes effect when multiple transactions can be executed concurrently on the same resources. That means concurrent transactions can not interfere one with another. Therefore each concurrent execution on the shared resource must be equivalent to some serial ordering of contained transactions which is why the isolation is often also referenced as serializability.

From the perspective of an external user the isolation property means that the transaction appears as it was executed entirely by itself. This means that even if there are multiple transactions in the system executed concurrently, this fact is hidden from the every external view.

As an instinctive extension of the consistency property, the serial execution of the transaction keeps the consistent state. The execution of the transactions in parallel therefore cannot result into the inconsistent system.

\subsubsection{Isolation levels}

As transactions were first defined only in the database environment, in practice we distinguish several levels that describe to which extent the isolation succeeds.


\subsection{Durability}



\section{Transaction models}

\url{https://ress.infoq.com/minibooks/JTDS/en/pdf/javatransactionsbook.pdf?Expires=1512505329&Signature=NBftdwZSbBAKvAxQ1LfPMzvO~9vusCCd1WxWIiqcSU7Ja-JB7LtolrFSCIBkZM2mDYNc6Zw9erfuo-31FosKWWx64~tUEPbxBD781bVplO3Maal3Bji8wMnFvROJucBH60m3-WzHJTlCJN7OxJ6l1ciBJNlSGY4HSc~QzSjEbT8_&Key-Pair-Id=APKAIMZVI7QH4C5YKH6Q}


\section{Distributed transactions}

A distributed transaction is the transaction performed in a distributed system.  
The distributed system consists of a number of independent devices connected through a communication network. Such systems are liable to the frequent failures of individual participants or communication channels between them. 

The transaction manager can be implemented as a separate service or being placed with some participant or the client. \textbf{TODO}

\section{Transaction manager}

Every transaction is associated with a transaction coordinator or transaction manager which is responsible for the control and supervision of the participants performing individual operations. It is a component liable for coordinating transactions in the sequential or parallel execution across one or more resources. It provides proper and complete execution and it administers the comprehensive result of the transaction.  Applications are commonly required only to contact the transaction manager about the start of the transaction.

The main responsibilities of transaction manager are starting and ending (commit or abort) of the transaction, management of the transaction context, supervision of transactions scoped across multiple resources and the recovery from failure.

\subsection{Local transaction manager}

A local transaction manager or a resource manager is responsible for the coordination of transactions concerning only a single resource. Because of its scope it is often build in directly to the resource. The span of the resource is defined by its managing platform. 

The resource manager is required to provide a support for the participation in the global transactions that span over several resources. This means that it is effectively capable to handle complete transaction processing to the different transaction manager.

\section{Failure handling}





CA CP theorem - CAP




\chapter{Microservices architecture pattern}

This chapter introduces the basic concept of microservices and describes why modern scalable enterprise applications should be developed implementing this pattern.

\section{Architecture pattern}

Microservices as a subset of service oriented architecture (SOA) \cite{soa} is an architectural pattern which offers an intuitive approach to common problems following the software development. Instead of SOA which builds the applications around the logical domain, microservices are built around the business model. Each microservice represents the part of the system.

\subsection{Monolithic architecture}

When describing microservices, the best way is to start by defining its opposite, monolithic architecture. When the application is developed in a monolithic fashion, all of its content is being implemented and deployed as a single archive. Every component, i. e. "a unit of software that is independently replaceable and upgradeable" \cite{microservices}, is tightly coupled with the application, which is using it. Because of the easy development, scalability and deployment of monolithic software this approach is being preferred by the majority of modern enterprise applications. However, when the application needs to be expanded, it can be difficult to maintain. For instance, even because of the minor change or update in the single component, the scaling and continuous deployment of the whole application can stagnate. 

\subsection{Microservice architecture}

Microservices introduced the application separation into  the self-maintained units – services \cite{intro_to_microservices}. The service is a single scalable and deployable unit, which is not dependent on any context. This means that the service can be maintained apart from applications which use it. In addition, microservices are being developed independently from other services. Each instance is managing its resources and is not able to directly access resources of any other service. This allows each request to data to be processed by the managing service. Service corresponds to component in monolithic architecture.

One of the biggest advantages of microservices is the ability to be deployed to the server, not affecting other applications or services. This allows separated teams to develop and maintain services independently. Applications based on this architecture utilize services by remote procedure calls instead of in-process calls used in monolithic architecture.

\section{Principles of microservices}

This section is inspired by the talk delivered by Sam Newman \cite{principles_of_microservices} in 2015 on the Devoxx conference in Belgium. In this presentation he described microservices from the business perspective. By his definition microservices are "Small autonomous services that work together" and they are based on eight principles.

\begin{enumerate}
	
	\item \textbf{Modeled around the business domain} -- As was stated in the beginning of this chapter, microservices as well as the teams which are maintaining them correspond to the business model. This means that the requirements on their functionality do not change frequently. This architecture scales applications vertically -- changes processed in one microservice do not affect other services and the system itself. They allow developers to focus on the particular part of the system rather than some specific technology. 
	
	\item \textbf{Automation} -- The services are managed by teams. The team is responsible for development, administration, deployment and the life cycle of the service infrastructure. When the number of services is small it is possible to maintain them manually but when their number increases, this will became unacceptable. Automation processes like testing or continuous delivery then allow the enterprises to scale more efficiently and speed up the mechanism of service coordination.
	
	\item \textbf{Hide the implementation details} -- Microservices use external resources. Often, there is a requirement to share the same resource between two or more services. One possibility to do this is by providing the resource directly. The problem with this approach is that when one service changes the resource other services need to react to the change. The idea of this principle is that each microservice maintains its own resources. It  provides the API\footnote{Application Programming Interface} to access them. This allows the developers to decide what is hidden. Every request for the data must be processed through the public interface.
	
	\item \textbf{Decentralization} -- Microservice architecture is build around the idea of self-sustaining development. Services are maintained autonomously. When the teams are not dependent on each other, they can work more freely which allows faster improvement. This principle also corresponds to partitioning of responsibilities. It accentuates that relevant business logic should be kept in the services themselves and the communication between them must be as simple as possible.  
	
	\item \textbf{Independent deployments} -- This is the most important principle of this architecture. It expands the base provided by the option of independent development. When the service is being deployed it should be the requirement that it cannot influence the lifespan of any other service. To achieve this, various techniques like consumer-driven contracts or co-existing endpoints can be used. Consumer-driven contracts make services to state their explicit expectations. These requirements are supported by the provided test suite for individual parts of the domain and they are run with each CI\footnote{Continuous integration} build. Co-existing endpoints model describes the situation when customers need to upgrade to the new version of the service. As customers cannot be forced to upgrade at the same time as the release happens, the idea is to make new endpoint which would process the client requests. Customers use both endpoints depending on the version their applications require. This allows them to upgrade. When the endpoint is no longer in use, it can be safely removed.
	
	\item \textbf{Customer first} -- Services exist to be called. It is indispensable to make these calls as simple as possible for the customers. For the developers it can be useful to have any feedback from the clients that use their service. The understanding of the API can be supported by a good documentation provided by API frameworks like Swagger \cite{swagger}, or by the service discovery to propagate services and make the discovery of service providers easier. To combine this information we can use the humane registries \cite{humane_registry} which indicate the human interaction. 
	
	\item \textbf{Failure isolation} -- Even if microservices force distributed development it is not a necessity that the failure of one service cannot influence another. This principle describes the distribution of resources to avoid the single point of failure. As there are many vulnerabilities in applications which can break, there is no precise manual on how to attain this principle.
	
	\item \textbf{High observability} -- Monitoring is an important part of development. As the microservice architecture is distributed it can be complicated to process this information. To make it more accessible aggregation is essential. Storing all logging entries and statistics in one place can highly impact the monitoring process. Another relevant point is to track the service calls as the services typically communicate with other services. By logging this information we can ensure traceability in case of failure.
	
\end{enumerate}



\chapter{Communication patterns}

\section{Consensus protocols}

\subsection{2PC}

\subsection{3PC raft paxos}

\section{Event based protocols}

Eventual consistency

\subsection{CQRS}



\chapter{Transactions in distributed environment}

\section{Saga pattern}

A saga, as described in the original publication \cite{sagas_publ}, is a long lived transaction that can be written as a sequence of transactions that can be interleaved with other transactions. Each operation that is a part of the saga represents an unit of work that can be undone by the compensation action. The saga guarantees that either all operations complete successfully, or the corresponding compensation actions are run for all executed operations to cancel the partial processing.

\subsection{Participants}


\chapter{Conclusion}



\makeatletter\thesis@blocks@clear\makeatother
\phantomsection %% Print the index and insert it into the
\addcontentsline{toc}{chapter}{Bibliography} %% table of contents.
\printindex

\bibliographystyle{IEEEtran}
\bibliography{IEEEabrv,references}

\appendix %% Start the appendices.

\end{document}
